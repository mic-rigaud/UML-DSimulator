\documentclass[a4paper, 11pt, oneside, oldfontcommands]{memoir}

%%%%% Packages %%%%%
\usepackage{lmodern}
\usepackage{palatino}
\usepackage[T1]{fontenc}
\usepackage[utf8]{inputenc}
\usepackage[english]{babel}


%%%%%%%%%%%%%%%%%%%%  PACKAGE SECONDAIRE

%\usepackage{amstext,amsmath,amssymb,amsfonts} % package math
%\usepackage{multirow,colortbl}	% to use multirow and ?
%\usepackage{xspace,varioref}
\usepackage[linktoc=all, hidelinks]{hyperref}			% permet d'utiliser les liens hyper textes
\usepackage{float}				% permet d ajouter d autre fonction au floatant
%\usepackage{wrapfig}			% permet d avoir des image avec texte coulant a cote
%\usepackage{fancyhdr}			% permet d inserer des choses en haut et en bas de chaque page
\usepackage{microtype}			% permet d ameliorer l apparence du texte
\usepackage[explicit]{titlesec}	% permet de modifier les titres
\usepackage{graphics}			% permet d utiliser les graphiques
\graphicspath{{./images/}}		% to say where are image
%\usepackage{eso-pic} 			% to put figure in the background
\usepackage[svgnames]{xcolor}	% permet d avoir plus de 300 couleur predefini
%\usepackage{array}				% permet d ajouter des option dans les tableaux
%\usepackage{listings}			% permet d ajouter des ligne de code
%\usepackage{tikz}				% to draw figure
%\usepackage{appendix}			% permet de faire les index
%\usepackage{makeidx}			% permet de creer les index
%\usepackage{fancyvrb}			% to use Verbatim
%\usepackage{framed}				% permet de faire des environnement cadre
%\usepackage{fancybox}			% permet de realiser les cadres
\usepackage{titletoc}			% permet de modifier les titres
\usepackage{caption}
\usepackage[a4paper, top=2cm, bottom=2cm]{geometry}
\usepackage{frbib}                      %permet d avoir une biblio francaise
\usepackage[babel=true]{csquotes}
\usepackage{colortbl}
\usepackage{listings}
\usepackage{xspace}
\usepackage{itemsep}

%\usepackage{graphicx}
\RequirePackage{pageGardeEnsta}	% permet d avoir la page de garde ensta

%\setsecnumdepth{subsection}
\setcounter{secnumdepth}{3}		% permet d'augmenter la numerotation
%\setcounter{tocdepth}{3}		% permet d'augmenter la numerotation

%%%%%%%%%%%%%%%%%%  DEFINITION DES BOITES
% Definition de couleur supplementaire
\definecolor{colString}{rgb}{0.6,0.1,0.1}

% Definition du langage
\lstdefinelanguage{LangageConsole}{%
    morekeywords={%
        time% mot-clé ``ligne''
    },
    morestring=[b]",
    morecomment=[l]{//},
    morecomment=[s]{/*}{*/},
}

% Definition du style
\lstdefinestyle{styleLangage}{%
    language        = LangageConsole,%
    basicstyle      = \bf\footnotesize\ttfamily\color{white},% ecriture standard
    identifierstyle = \color{white},%
    commentstyle    = \color{green},%
    keywordstyle    = \color{blue},%
    stringstyle     = \color{colString},%
    extendedchars   = true,% permet d'avoir des accents dans le code
    tabsize         = 2,%
    showspaces      = false,%
    showstringspaces = false,%
    numbers=left,%
    numberstyle=\tiny\ttfamily\color{black},%
    breaklines=true,%
    breakautoindent=true,%
        backgroundcolor=\color{black},%
}

\lstset{%
    style = styleLangage%
}

%%% JSON Console
\newcommand\JSONnumbervaluestyle{\color{blue}}
\newcommand\JSONstringvaluestyle{\color{red}}

% switch used as state variable
\newif\ifcolonfoundonthisline

\makeatletter

\lstdefinestyle{json}
{
  showstringspaces    = false,
  keywords            = {false,true},
  alsoletter          = 0123456789.,
  morestring          = [s]{"}{"},
  stringstyle         = \ifcolonfoundonthisline\JSONstringvaluestyle\fi,
  MoreSelectCharTable =%
    \lst@DefSaveDef{`:}\colon@json{\processColon@json},
  basicstyle          = \ttfamily,
  keywordstyle        = \ttfamily\bfseries,
  backgroundcolor     = \color{white},
  identifierstyle     = \color{black},
}

% flip the switch if a colon is found in Pmode
\newcommand\processColon@json{%
  \colon@json%
  \ifnum\lst@mode=\lst@Pmode%
    \global\colonfoundonthislinetrue%
  \fi
}

\lst@AddToHook{Output}{%
  \ifcolonfoundonthisline%
    \ifnum\lst@mode=\lst@Pmode%
      \def\lst@thestyle{\JSONnumbervaluestyle}%
    \fi
  \fi
  %override by keyword style if a keyword is detected!
  \lsthk@DetectKeywords%
}

% reset the switch at the end of line
\lst@AddToHook{EOL}%
  {\global\colonfoundonthislinefalse}

%%%%%%%%%%%%%%%%%%%%

\newcounter{rem}[chapter]

\newcommand{\remarque}[1]{\stepcounter{rem}\noindent\fcolorbox{OliveDrab}{white}{\parbox{\textwidth}{\textcolor{OliveDrab}{
\textbf{Remarque~\thechapter.\therem~:}}\\#1}}}

\newcounter{th}[chapter]

\newcommand{\theoreme}[2]{\noindent\fcolorbox{FireBrick}{white}{\stepcounter{th}
\parbox{\textwidth}{\textbf{\textcolor{FireBrick}{Théorème~\thechapter.\theth~:}}{\hfill \textit{#1}}\\#2}}}

\newcommand{\attention}[1]{\noindent\fcolorbox{white}{white}{\parbox{\textwidth}{\textcolor{FireBrick}{
\textbf{Attention !}}\\\textit{#1}\\}}}
%%%%%%%%%%%%%%%%%%%%%%%%%%%%%%%%%%%%%%%%%%%%%%%%%%%%%%%%%%%%%%%%%%%%%%%%%


%% INDEX %%%%%%%%%%%%%%%%%%%%%%%%%%%%%%%%%%%%%%%%%%%%%%%%%%%%
\makeindex

%%%%% Useful macros %%%%%
\newcommand{\latinloc}[1]{\ifx\undefined\lncs\relax\emph{#1}\else\textrm{#1}\fi\xspace}
\newcommand{\etc}{\latinloc{etc}}
\newcommand{\eg}{\latinloc{e.g.}}
\newcommand{\ie}{\latinloc{i.e.}}
\newcommand{\cad}{c'est-à-dire\xspace}
\newcommand{\st}{\ensuremath{\text{\xspace s.t.\xspace}}}
\newcommand{\umld}{UML Designer\xspace}

%%%% Definition des couleur %%%%

\newcommand\couleurb[1]{\textcolor{SteelBlue}{#1}}
\newcommand\couleurr[1]{\textcolor{DarkRed}{#1}}


%% number page style style %%%%%%%%%%%%%%%%%%%%%%%%%%%%%%%%%%%%%%%%%%%%%%%%%%%%%%

\pagestyle{plain}
%\pagestyle{empty}
%\pagestyle{headings}
%\pagestyle{myheadings}



%% chapters style %%%%%%%%%%%%%%%%%%%%%%%%%%%%%%%%%%%%%%%%%%%%%%%%%%%%%%
%% You may try several styles (see more in the memoir manual).

%\chapterstyle{veelo}
%\chapterstyle{chappell}
%\chapterstyle{ell}
%\chapterstyle{ger}
%\chapterstyle{pedersen}
%\chapterstyle{verville}
\chapterstyle{madsen}
%\chapterstyle{thatcher}


%%%%% Report Title %%%%%
\title{UML-DSimulator}
\author{\textsc{IETA Michaël Rigaud }}
\date{\today}
\doctype{SPID TSe}
\promo{promo 2017}
\etablissement{\textsc{Ensta} Bretagne\\2, rue François Verny\\
  29806 \textsc{Brest} cedex\\\textsc{France}\\Tel +33 (0)2 98 34 88 00\\ \url{www.ensta-bretagne.fr}}
\logoEcole{\includegraphics[height=4.2cm]{logo_ENSTA_Bretagne_Vertical_CMJN}}
\respo{Prof. Hans Vangheluwe}
\tuteur{Simon Van Mierlo}
\addto\captionsenglish{% Replace "english" with the language you use
  \renewcommand{\contentsname}%
    {Table of contents}%
}



%%%%%%%%%%%%%%%%%% DEBUT DU DOCUMENT
\begin{document}

\maketitle
\thispagestyle{empty}
\newpage

\chapter*{Abstract}
\addcontentsline{toc}{chapter}{Abstract}

UML-DSimulator is an open source\footnote{under GNU GPL license} plugin which add a simulator to \umld. This plugin has been entirely produce during this internship, but it is based on a simulator developed by Ciprian Teodorov.



%%% Local Variables:
%%% mode: latex
%%% TeX-master:  "../rapport_de_base"
%%% End:


\chapter*{Acknowledgement}
\addcontentsline{toc}{chapter}{Acknowledgement}
First of all, I would like to express my deep gratitude to Professor Hans Vangheluwe and Mr Simon Van Mierlo, my research supervisors, for their patient guidance, enthusiastic encouragement and useful critiques of this research work. I would also like to thank Professor Champeau, for his advice and for giving me the opportunity to do this internship. My grateful thanks are also extended to Professor Teodorov for his help in the manipulation of his simulator.

I would also like to extend my thanks to all members of the MSDL laboratory for their help in offering me the resources in running the program, and their welcome. And more particularly to Yentl Van Tendeloo for his technical help and Claudio Gonçalves Gomes for taking time to read my report.

%Finally, I wish to thank my parents for their support and encouragement throughout my study.

\newpage

%%% Local Variables:
%%% mode: latex
%%% TeX-master: "../rapport_de_base"
%%% End:


\tableofcontents


%%%%%%%%%%%%%%%%% INTRODUCTION

\chapter*{Introduction}
\addcontentsline{toc}{chapter}{Introduction}

During my second year school at ENSTA Bretagne, Mr Champeau taught us UML Diagrams. During this lesson, He shown us the possibility to create Codes from UML Diagram and the possibility to simulate UML Diagrams such as an overview of the running. But to do that, He needed a tool to create UML Model and simulate them. The two more user-friendly tools which permit that are: Rhapsody and Papyrus.

Papyrus use Moka to simulate UML Model and it was not well adapted for his lesson, so he choose Rhapsody. However, problems are that Rhapsody is not an open source software, it is only for Windows OS, and it is not free. That is why many student said that you won't use this software outside the lesson.

Mr Champeau has proposed this internship to fill in the lack of simulator in open source UML Modelers.

This report is going to present the evolution of the project during this internship. To begin, I will present the host structure. Then, I will explain the situation before the beginning of the project, so issues, aims, and the tools choice for preserve the organization of the project. After, I will show the technical choice given and the result of the technical part. To finish, I will highlight the contribution of this internship for my professional project.

\newpage
%%%%%%%%%%%%%%%%%%%%%%%%

\part{Presentation of the context}


\chapter{Presentation of the project}

\section{The goal}

The goal of this project is to create a simulator of Statechart which can be use with UMLDesigner. This simulator should permit to visualize and debug a model of a state machine. Moreover, UMLDesigner is a modeling software for UML model and Statechart, so we could create the model and simulate it on the same tools. The picture \ref{fig:project} represent the aim of this project.

\begin{figure}[h]
  \centering
\includegraphics[width=\textwidth]{project}  
  \caption{Description of the project}
  \label{fig:project}
\end{figure}


\section{Tools at the disposal}

At the begin of this project, some of the tools, which were needed, existed. In fact, ULMDesigner is a UML modeling tool develop by \textit{Obeo}. However, it didn't exist yet a simulator for Statechart adapted for UMLDesigner. On the chapter \ref{chap:UMLDesigner}, the running of UMLDesigner will be discuss.~\\

Then, Mr Ciprian Theodorov, one of my professor, has developed a simulator for Statechart. This simulator needed to be improved, but it composed a good beginning for this project. 



%%% Local Variables: 
%%% mode: latex
%%% TeX-master: "../rapport_de_base"
%%% End: 



\part{Presentation of the project}


\chapter{Goals}
\label{chap:issues}


As it was explained in the introduction, the aim of this project is to propose a visualization of UML Model for \umld.

\section{Short term goals}

In the short term, this plugin should permit Mr. Champeau and Mr. Teodorov to use \umld in the classroom, as a free and portable alternative to Rhapsody. In this way, every student will be able to install on their own computer the tool use in classroom.

Advantages of \umld when compared to Rhapsody: free, open source, works on Windows, Linux, and Apple.

\section{Long term goals}

Because this plugin is open source and downloadable on a Git Server\footnote{\url{https://msdl.uantwerpen.be/git/michael/UML-DSimulator}}, it can be used by anybody. My hope is that this plugin can be improved by the community, student, teacher, \etc... and becomes a serious alternative to Rhapsody and Papyrus simulator.

~\\

  \begin{figure}[h]
    \begin{minipage}{0.45\linewidth}
      \centering
      \includegraphics[width=0.7\textwidth]{Rapsody}
      \caption{Rational Rhapsody}
      \label{fig:rhapsody}
    \end{minipage}\hfill
    \begin{minipage}{0.45\linewidth}
      \centering
      \includegraphics[width=\textwidth]{Papyrus}
      \caption{Papyrus}
      \label{fig:papyrus}
    \end{minipage}
  \end{figure}


%%% Local Variables:
%%% mode: latex
%%% TeX-master: "../rapport_de_base"
%%% End:

\chapter{The plans of the project}
\label{chap:goals}

\section{Definition of my project}

After some interview with Mr. Champeau, I define the table of requirements (tabular \ref{tab:requirements}). This table contains all requirements define by the client to respect the issues of this project define before.

\textit{FS} means service function and \textit{C} means constraints.

\noindent{}
\begin{table}[!h]
  \centering
  \begin{tabular}[h]{|m{0.2\linewidth}|m{0.25\linewidth}|m{0.45\linewidth}|}
    \hline
    \multicolumn{3}{|c|}{Table of requirements}\\
    \hline
    Number&Type of Designation&Designation\\
    \hline
    FS1&UI&Represent the visualization in the UML Model with graphical tools\\
    \hline
    FS2&UML Designer&Works by default this the Ciprian simulator\\
    \hline
    FS3&Simulator&Have the possibility to change the simulator with an other simulator, for example SCCD\\
    \hline
    FS4&Simulation&Give some debugging tools, for example play, stop, \etc...\\
    \hline
    C1&License&Open Source\\
    \hline
    C2&Compatibility&Be multiplatform. Works on Linux, Windows, and Apple.\\
    \hline
    C3&Documentation&Have documentation, code readable, modularity, etc. In that way, this project could be improved during another internship. \\
    \hline
    C4&User&Be user-friendly\\
    \hline
  \end{tabular}
  \caption{Table of requirements}
  \label{tab:requirements}
\end{table}

Then, it is also possible to represent the project with a octopus diagram (figure \ref{fig:octopus}). This diagram permit to show the interaction with the outside world that the client expected.

\begin{figure}[!h]
  \centering
  \includegraphics[width=\textwidth]{pieuvre}
  \caption{Octopus diagram}
  \label{fig:octopus}
\end{figure}

\newpage
\section{Goals}

As it was explain in the introduction, the main goal of this project is integrate a simulator in UML Designer. To do that Mr. Ciprian put at my disposal his own Simulator. So it is possible to describe my goals in this order:

\begin{enumerate}
\item Find the way to add plugin in \umld, and understand how it is possible to add some feature.
\item Understand how work the Ciprian simulator.
\item Find a way to integrate the simulator but keep the possibility to change it. Moreover it should be kept in mind that the simulator was not finish so the integration of the simulator need to preserve modularity.
\item Propose some debugger tools. For example: a play button, a stop button, \etc.
\item Write documentation and comment in the code to be reusable. Mr Champeau wanted to keep the choice to do some improvement after the end of this internship.
\item Try an other simulator and compare its performance with the Ciprian simulator.
\end{enumerate}





% \section{Solution proposed}

% After this definition of the goals and the requirements,

% The main goal of this project is to visualize a simulation of Statechart in UML Designer. The simulator should permit to visualize and debug a model of a state machine. Moreover, UML Designer is a modeling software for UML model and Statechart, so we could create the model and simulate it on the same tools.


% The picture \ref{fig:project} represent the aim of this project.

% \begin{figure}[h]
%   \centering
%   \includegraphics[width=\textwidth]{project}
%   \caption{Description of the project}
%   \label{fig:project}
% \end{figure}







\section{Tools at my disposal}

It is a short description of \umld and the Ciprian Simulator. A further description can be found in the appendix \ref{chap:UMLDesigner} and \ref{chap:simulator}.
\newpage
\subsection{\umld}
\begin{figure}[!h]
  \begin{minipage}[h]{0.45\linewidth}
    \centering
    \includegraphics[width=0.6\textwidth]{logo}
    \caption{UML Designer logo}
    \label{fig:logo}
  \end{minipage}\hfill
  \begin{minipage}[h]{0.45\linewidth}

    At the beginning of this project, some tools were at my disposal. First of all, it was \umld created by the french company \textit{Obeo}. It is an open source software documented so I could download the source and work on it. It is a UML modeler with a user interface. It is based on Eclipse and Sirius. It follows the UML2 standard which is know and documented.

  \end{minipage}
\end{figure}


\subsection{Simulator}
\begin{figure}[!h]
  \begin{minipage}[h]{0.45\linewidth}

    Then, Mr Ciprian Teodorov, one of my professor, has developed a simulator for UML Model. This simulator needed to be improved, but it composed a good beginning for this project.

  \end{minipage}\hfill
  \begin{minipage}[h]{0.45\linewidth}
    \centering
    \includegraphics[width=0.9\textwidth]{simulator}
    \caption{Mr Teodorov simulator}
    \label{fig:sim}
  \end{minipage}
\end{figure}







%%% Local Variables:
%%% mode: latex
%%% TeX-master: "../rapport_de_base"
%%% End:


\chapter{Organisation of the work}

\section{Calendar}


\begin{tabular*}{1\textwidth}{@{\extracolsep{\fill}} |c|*{14}{c|}}
\hline
  Tasks/weeks & 1 &2 &3&4&5&6&7&8&9&10&11&12&13&14\\
\hline
State of the art&-&-&&&&&&&&&&&&\\
\hline
Create a plugin&&&-&&&&&&&&&&&\\
\hline
Visualize the simulation&&&&-&-&-&&&&&&&&\\
\hline
Unit tests&&&&&&-&-&&&&&&&\\
\hline
Integration tests&&&&&&&&-&&&&&&\\
\hline
Improve the simulator&&&&&&&&&-&-&&&&\\
\hline
Try an other simulator&&&&&&&&&&&-&-&&\\
\hline
Redaction&&-&-&-&-&-&-&-&-&-&-&-&-&\\
\hline
Soutenance&&&&&&&&&&&&&&-\\
\hline
\end{tabular*}

\section{Tools use for the project}

The Framaboard application:

\begin{figure}[h]
  \centering
  \includegraphics[width=\textwidth]{framaboard}
  \caption{Screen shot of the framaboard}
  \label{fig:framaboard}
\end{figure}


The web site of MSDL researcher:

\begin{figure}[h]
  \centering
  \includegraphics[width=\textwidth]{msdl}
  \caption{MSDL web site}
  \label{fig:msdl}
\end{figure}


%%% Local Variables: 
%%% mode: latex
%%% TeX-master: "../rapport_de_base"
%%% End: 


\part{Results of the internship}


\chapter{Communication inter process}

\section{Type of communication conceivable}

A lot of type of communication inter process were suggested to create a discussion enter the plugin and the simulator. But we will present only the most consistent.

The communication is the part the most important of this project, because that will implement the interface between the two software.

\subsection{Socket}

\begin{tabular}{|p{0.45\textwidth}||p{0.45\textwidth}|}
\hline
  \textbf{Advantages}&\textbf{Drawback}\\
\hline
Work with every simulator type (python, java, ...) & communication synchronous\\
\hline
\end{tabular}

\subsection{File}

\begin{tabular}{|p{0.45\textwidth}||p{0.45\textwidth}|}
\hline
  \textbf{Advantages}&\textbf{Drawback}\\
\hline
Problem when two software want to change the same file at the same moment& Communication asynchronous\\
\hline
\end{tabular}

\subsection{Named pipe}

\begin{tabular}{|p{0.45\textwidth}||p{0.45\textwidth}|}
\hline
  \textbf{Advantages}&\textbf{Drawback}\\
\hline
It is possible to use the Simulator outside the graphical modeling tool & \\
\hline
\end{tabular}


\subsection{Shared Memory}

\begin{tabular}{|p{0.45\textwidth}||p{0.45\textwidth}|}
\hline
  \textbf{Advantages}&\textbf{Drawback}\\
\hline
It is possible to use the Simulator outside the graphical modeling tool & \\
\hline
\end{tabular}


\subsection{Thread}

\begin{tabular}{|p{0.45\textwidth}||p{0.45\textwidth}|}
\hline
  \textbf{Advantages}&\textbf{Drawback}\\
\hline
&Need to recode the simulator \\
\hline
\end{tabular}


\subsection{Our solution}

The solution was not in this list of common way to communicate inter process. In fact, we use the \textit{Runtime} class which is in the java library.~\\


\begin{tabular}{|p{0.45\textwidth}||p{0.45\textwidth}|}
\hline
  \textbf{Advantages}&\textbf{Drawback}\\
\hline
It is possible to use the Simulator outside the graphical modeling tool & \\
\hline
Work with every type of simulator& \\
\hline
\end{tabular}





%%% Local Variables: 
%%% mode: latex
%%% TeX-master: "../rapport_de_base"
%%% End: 


\chapter{The simulator}
\label{chap:simul}

Afterwards, I worked on the simulator to understand how it works and implement the layer of communication.

\section{Specificity of the UML Model}

The Ciprian simulator simulates a UML model but this UML model need to have some specificities in the architecture and in the language.

\umld use 2 files to save a uml project. The first is named \textit{model.uml} it contains all UML elements and the declaration of UML diagrams. The second is named \textit{representation.aird}, it contains the specification of the graphical representation, which consists of the positions, colors, and other graphical attributes of the elements described in model.uml file.

To work, the simulator needs the \textit{model.uml} file. Moreover, this file need to contain some specifics features. It needs a class \textbf{SUS} which contains the declaration of all other classes and all other classes need to have a State Machine diagram associated, as you can see on the Figure \ref{fig:simulateur}.

\begin{figure}[h!]
  \centering
  \includegraphics[width=\textwidth]{simulation}
  \caption{Representation of the most important elements for the simulator}
  \label{fig:simulateur}
\end{figure}

\section{Additions to Teodorov Simulator}

Then, I made some research on the code of the simulator. I realized that the simulator was not develop to communicate and notify any changing. So I changed the code of the simulator to add an Observer pattern at the class \textit{SimulationModel} because this class is a controller of the simulator.

\begin{quotation}
<<The Observer pattern is a software design pattern in which an object, called the subject, maintains a list of its dependents, called observers, and notifies them automatically of any state changes, usually by calling one of their methods.>> \cite{wiki_pattern}
\end{quotation}


I used this pattern to notify the communication class of a changing.

\section{Communication}



The communication Layer was written in Java. As it was explain in the chapter before, the simulator and its layer were implemented, without loss of generality, inside the plugin. The main class of the plugin initiates the Simulator in a new thread and then there is only a communication enter the plugin and the simulator though the localhost loop by socket.

%In the chapter before, we have decided to put the simulator outside. However, . So I put the simulator and its layer in . However,

Moreover, to be sure that it is possible to change the default simulator with an outside simulator I did some test where the thread is initiated by an other software, and that worked. % So for the rest of this report we consider that the simulator is outside the plugin because the comportment is the same.%, but the default simulator is in fact initiate by the plugin.






%%% Local Variables:
%%% mode: latex
%%% TeX-master: "../rapport_de_base"
%%% End:


\chapter{Plugin}
\label{chap:results}


\section{How to write a plugin}


The first thing that I looked for in the \umld documentation was how add something in the software. It is a real challenge in software development. Fortunately, I quickly found on the \umld developer Guide, that there is a the developer environment to do that. This environment was an Eclipse environment, which is looking as the figure \ref{fig:eclipse}.

\begin{figure}[h]
  \centering
  \includegraphics[width=0.8\linewidth]{eclipse}
  \caption{Eclipse environment}
  \label{fig:eclipse}
\end{figure}

Then, I use the fact that \umld is based on Eclipse to create the plugin. The creation of a plugin is the same as an Eclipse plugin. So I made some research on how write an Eclipse plugin. I learn that there is a special structure. This structure permit to keep the modularity and the possibility to install the plugin in all platform. The structure of an eclipse plugin follow the structure show on the figure \ref{fig:plugin}.

\begin{figure}[h]
  \centering
  \includegraphics[width=0.5\linewidth]{structure_plugin}
  \caption{Structure of an eclipse plugin}
  \label{fig:plugin}
\end{figure}

\noitemsep
\begin{itemize}
\item The directory \textit{JRE System Library} contains all libraries used to run the plugin
\item The directory \textit{src} contains sources of the plugin
\item The file \textit{MANIFEST.MF} contains addition information for the integration of the plugin
\item The file \textit{build.properties} contains information to export the project
\end{itemize}
\doitemsep

\section{General presentation}

During the development of the plugin, I tried to separate elements of the \umld API, the elements of the UI, and the communication Layer.

The architecture of my plugin follow the UML class diagram presented on the figure \ref{fig:classDiagram}. However, all link enter classes are not represented on it to not overloaded the model. A short description of all sub package:

\noitemsep
\begin{description}
\item[MainView] It is the main class, it is the first class call by \umld when it load the plugin
\item[features] This package contains all graphical elements of the plugins
\item[communication] This package contains the layer of communication of the plugin
\item[tools] This package contains some tools for all class
\item[design] This package contains all classes to change the appearance of the UML Model
\item[model] This package contain a class with all information of the status of the model during the simulation
\end{description}
\doitemsep

Then, it is possible to underline that I use a \textit{Observer} pattern enter classes \textit{MainView} and \textit{CommunicationP} as with the Simulator. This pattern permit to actualize the view of the model when a new communication is received.

\begin{figure}[h]
  \centering
  \includegraphics[width=\linewidth]{umlClassDiagram}
  \caption{plugin class diagram}
  \label{fig:classDiagram}
\end{figure}



\section{Functionality implemented}

During this project I had time to implements a lot of functionality for the simulation.


On the figure \ref{fig:result}, you can see the result of my project. It is an Eclipse view, on the top there is a list of all transition possible, and on the bottom it is a tree of the project view. The tree has at the top the name of the project simulated, then the name of all class, end to finish the name of all instances.

\begin{figure}[h]
  \centering
  \includegraphics[width=0.4\linewidth]{result}
  \caption{result in \umld}
  \label{fig:result}
\end{figure}

On the top, if the user click on a transition this transition is selected as the next transition, On the bottom, if he click on an instance of a class it is chosen as the visible instance of the model, and if he click on the class name all instances of this class are visible (useful only if there is multiple instance for a same class). Current instances visible are in red.
~\\

On the top, it is possible to see that there is many button. These buttons create some action:

\noitemsep
\begin{description}
\item[home] permit to change the project that we want to simulated
\item[play] launch a simulation where a random step is chosen every 1s.
\item[stop] stop the simulation launch with play
\item[replay] restart the simulation of the project
\item[menu] contain all previous buttons but also:
  \begin{description}
  \item[stop simulator] stop the communication with current simulator
  \item[wait a simulator] start a new server and wait a communication of a new simulator
  \item[restart simulator] restart a communication with the default simulator
  \end{description}
\end{description}
\doitemsep


To finish, I implemented the possibility to see the current state of all state machine diagram in the UML model. You can see on the figure \ref{fig:result2} a test that I did on one model given by Ciprian Teodorov. The red state is the current state.

\begin{figure}[h]
  \centering
  \includegraphics[width=\linewidth]{umldsimulator2}
  \caption{result of a simulation}
  \label{fig:result2}
\end{figure}


% I am going to present the result of my work. If you want a better description of the problems and the choice taken you can read the appendix ???.


% \section{Plugin}

% First I will talk about the plugin. The plugin is the part which realize the display in \umld and the communication to the layer of the simulator. \umld is based on Eclipse so to realize the plugin I realize a Eclipse plugin.
% ~\\



%%% Local Variables:
%%% mode: latex
%%% TeX-master: "../rapport_de_base"
%%% End:


\chapter{SCCD}
\label{chap:sccd}

In this chapter, I will talk about SCCD. In the MSDL laboratory, they use SCCD to simulate Statechart models. After a presentation of my project in July, they suggested me to try their simulator and compare it to the Ciprian simulator. It is possible to find more explanation about SCCD on the Master's thesis published by Glenn De Jonghe \cite{sccd} and on the Master's thesis published by Joeri Exelmans \cite{sccd2}.

\section{Short description}

SCCD\footnote{signified StateChart and Class Diagram} is an open source simulator and code generator of Statechart developed by the MSDL laboratory. An SCCD model consists of a class Diagram and a StateChart attached on each classes. It also uses some special semantics to instantiate classes with an object manager.

Because a state machine is a statechart\footnote{Statechart is an extension of state machine, invented by David Harel}, for the rest of this chapter we can consider that our models conform to the SCCD language.


\section{Transformer}

As we saw before, the project has been written to have a ability to change the simulator. However, the model needs to be written in scxml standard to be interpreted by SCCD. So the first things that I have to do is a transformer in XSLT. XSLT is a language for transforming XML documents into other XML documents.
~\\

After some research on the internet I found only one transformer written by apache on Github\cite{apache}. The last commit of this project was in 2009, so we can assume that the project is abandoned. I had tried to use it but it didn't work. For this reason I have created a new transformer, but I used some part of this project.
~\\

My transformer has the ability to transform a xmi file as a scxml file. However, model given by my professor had some specificities so I take care to adapt to its.

For example, when there is a script in ABCD language and the script is <<send eventA to itsPinger>> the translation in the state machine is <<raise eventA to itsPinger>>. Then I use the fact that our project always have a \textit{SUS} class which contains all other classes, so in the scxml model the \textit{SUS} class start all other classes.

\section{Utilization}

\subsection{SCCD debugger}

In my project I want to visualize states of all state machine. To do that, I need some informations of the status of the model. SCCD don't permit this type of utilization, but the SCCD debugger written by Simon Van Mierlo can do it.
~\\

However, during my internship this debugger wasn't finished. The debugger can create only one class because the object manager doesn't work.
~\\

To prove that my scxml model created automatically will work when the debugger will be finished, I tried to do a proof of concept. %I use my scxml model created by my transformer automatically in SCCD and I verify the running.

\subsection{SCCD}

To do this prove of contest, I achieve some tests on the pure SCCD. I use the last version of SCCD published in the beginning of august.


I quickly realize that my model had infinite loop. These loops are explain because in the model there is some state which have transition on itself, and this transitions are always verify. In the Ciprian simulator it was not a problem because the user has to choose the next transition and so he was the condition. To fix this problem I add on these transitions a timer of 1s.

The Figure \ref{fig:sccd_resume} resume the work done to execute xmi model with SCCD. You can also see on the Figure \ref{fig:sccd} the result of the Ping-Pong example on SCCD. As it was expected there is an event which is send from ping to pong and then from pong to ping.
\begin{figure}[h]
  \centering
  \includegraphics[width=\linewidth]{scxml}
  \caption{Resume}
  \label{fig:sccd_resume}
\end{figure}

\begin{figure}[h]
  \centering
  \includegraphics[width=\linewidth]{sccd}
  \caption{pingpong simulation on SCCD}
  \label{fig:sccd}
\end{figure}


%%% Local Variables:
%%% mode: latex
%%% TeX-master: "../rapport_de_base"
%%% End:


\chapter{Tests}
\label{chap:test}

\section{Unit tests}

During this project I did some unit tests to preserve the code during the development. But, I had a lot of difficulties to tests user interface features, so I chose, on Simon advice, to don't tests them. However, the architecture of the plugin clearly separates UI code from essentially functionalities, which where tested.
~\\

To do this unit tests, I used junit and an eclipse feature EclEmma which permit to see the coverage of code during unit tests. On the Figure \ref{fig:coverage}, you can see the result of the coverage show by EclEmma about my project.
~\\

First of all, you can see the package json was not well tested. This package was written by stleary\cite{json}, so I assumed it was already well tested.

Then, packages org.ensta.uml.sim.views.features and org.ensta.uml.sim.views.design only contain class which do action on the user interface. So I didn't tests them.

After this remarks, it is possible to notice that I tested more then 80\%\footnote{It is the usual value of acceptable coverage} of the code use in other classes.


\begin{figure}[h]
  \centering
  \includegraphics[width=\linewidth]{coverage}
  \caption{Coverage view of my project}
  \label{fig:coverage}
\end{figure}

\section{Integration tests}

I also did some integration tests to verify that the plugin runs in all major platforms (Windows, Linux, Apple).

During all my project I verified that my plugin could be use on my own computer without the eclipse developer environment. I have a Ubuntu 16.04 LTS.

Then, at the end of my project, I tried to use this plugin on other platform. To do that, I use virtual machine with VirtualBox. I tested on a Windows virtual machine with W7 (Figure \ref{fig:windows}), and a Kali virtual machine based on Debian (Figure \ref{fig:kali}). However, I didn't try on OSX because I didn't find a OSX machine.

\begin{figure}[h]
  \centering
  \includegraphics[width=\linewidth]{kali}
  \caption{Screen-shot of the kali virtual machine}
  \label{fig:kali}
\end{figure}

\begin{figure}[h]
  \centering
  \includegraphics[width=\linewidth]{windows}
  \caption{Screen-shot of the Windows virtual machine}
  \label{fig:windows}
\end{figure}


%%% Local Variables:
%%% mode: latex
%%% TeX-master: "../rapport_de_base"
%%% End:

% \input{Partie/UI}


\part{Contribution of this internship for my professional project}

\chapter{Contribution}




%%% Local Variables:
%%% mode: latex
%%% TeX-master: "../rapport_de_base"
%%% End:


%%%% CONCLUSION %%%%%%%%%

\chapter*{Conclusion}
\addcontentsline{toc}{chapter}{Conclusion}
\newpage
%%%% ANNEXE %%%%%%%%%%%%

\part*{Annex}
\addcontentsline{toc}{part}{Annexe}
\appendix
\chapter{\umld}
\label{chap:UMLDesigner}

\section{Description}

\umld is an open-source tool to edit and visualize UML2 models created by the French company:
\textit{Obeo}. The project is licensed under the EPL\footnote{Eclipse public license}

\begin{figure}[h] \centering
  \includegraphics[width=0.3\textwidth]{logo}
  \caption{UML Designer logo}
  \label{fig:logo}
\end{figure}

\section{Utilization}

\umld is a graphical modeling tool for UML2 as defined by OMG\footnote{Object Management Group\cite{omg}}. As
you can see on the figure \ref{fig:umldesigner}, it permit to create diagram on which ones it is
possible to add some elements. The type of the elements proposed depend on the types of the diagram
chosen. For example, if you choose a \textit{User case diagram} it is possible to add 'user'
component that is impossible in \textit{Class diagram}.

So with graphical action it is possible to create many UML diagram which have transverse elements.

To finish, it is possible to create the code of the application that you have develop from the
model.


\begin{figure}[h] \centering
  \includegraphics[width=0.8\textwidth]{umldesigner}
  \caption{Screen shot of \umld}
  \label{fig:umldesigner}
\end{figure}



\section{List of diagram supported}

\begin{itemize}
\item Packages diagram
\item Use case diagram
\item Activity diagram
\item Class diagram
\item Component diagram
\item Composite Structure diagram
\item Sequence diagram
\item State Machine diagram
\item Documentation table
\item Use Case cross table
\item Package containment diagram
\item Profile diagram
\end{itemize}



\section{Released}
\begin{center}

  \begin{tabular}{|c|c|}
    \hline
    \textbf{Version} & \textbf{Release Date} \\
    \hline
    1.0.0 &2012\\
    \hline
    2.0.0 &17 January 2013\\
    \hline
    2.1.0 &1 February 2013\\
    \hline
    2.2.0 &12 April 2013\\
    \hline
    2.3.0 &13 June 2013\\
    \hline
    2.4.0 &13 September 2013\\
    \hline
    3.0.0 &17 January 2014\\
    \hline
    4.0.0 &8 July 2014\\
    \hline
    4.0.1 &5 August 2014\\
    \hline
    5.0.0 &29 May 2015\\
    \hline
    \cellcolor{green}6.0.0 &19 October 2015\\
    \hline
  \end{tabular}

  \begin{tabular}{c}
    Legend:\\
    \cellcolor{green}Latest stable release\\
  \end{tabular}

\end{center}

\section{Base on}

\umld is based on a Eclipse and Sirius. It is a UML2 Eclipse plugin.

\subsection{Sirius}
Sirius is an open-source software project of the Eclipse Foundation. Sirius allows to create graphical modeling workbench. It include EMF\footnote{Eclipse Modeling Framework} and GMF\footnote{Graphical Modeling Framework}. On the figure \ref{fig:sirius}, it is possible to see the architecture of Sirius.


\begin{figure}[h]
  \centering
  \includegraphics[width=\textwidth]{sirius_archi}
  \caption{Sirius architecture\cite{sirius}}
  \label{fig:sirius}
\end{figure}

\subsection{Eclipse}

\umld is base on Eclipse.
The interface is the same as Eclipse. You can notice on figure
\ref{fig:umldesigner} that the menu are the same in the both software.



\begin{figure}[h] \centering
  \includegraphics[width=0.5\textwidth]{archi}
  \caption{The \umld kernel}
  \label{fig:kernel}
\end{figure}






%%% Local Variables:
%%% mode: latex
%%% TeX-master: "../rapport_de_base"
%%% End:


\chapter{Simulator}

\section{Description}


At the beginning of this project, we had at our disposal the simulator of Mr Teodorov (figure \ref{fig:sim}). This simulator have a graphic user interface as you can see on the figure \ref{fig:sim}.


\begin{figure}[h]
  \centering
  \includegraphics[width=0.5\textwidth]{simulator}
  \caption{Mr Teodorov simulator}
  \label{fig:sim}
\end{figure}


The simulator is compose on 4 part.
\begin{itemize}
\item On the top: some buttons to select an action
\item On the top-left-corner: The list of the next step
\item On the bottom-left-corner: The State Machine associated to the Current State.
\item On the right: A visualization of the Statechart
\end{itemize}

\section{Specificity of the uml file}

This simulator simulate a uml file. The uml file need to have a particular architecture.

\umld to save the uml project use 2 files. The first is named ``model.uml'' and the second is named ``representation.aird''.

To work, the simulator need the \textit{model.uml} file. Moreover, this file need to contain some specifics feature. It need a class \textbf{SUS} which contain the declaration of all other classes and all other classes need to have a State Machine diagram associated. You can see on the figure \ref{fig:simulateur}, that all classes need to have their own State Machine diagrams.

\begin{figure}[h!]
  \centering
  \includegraphics[width=\textwidth]{simulation}
  \caption{representation of the most important elements of the simulator}
  \label{fig:simulateur}
\end{figure}




%%% Local Variables:
%%% mode: latex
%%% TeX-master: "../rapport_de_base"
%%% End:


\chapter{Inter-process communication}
\label{annex:choice}


We are only interesting in communication which permit to exchange data and synchronize them. There is a short presentation of all other communication considered.

\section{Message queue}

\begin{tabular}{|p{0.45\textwidth}||p{0.45\textwidth}|}
\hline
  \textbf{Advantages} & \textbf{Drawback}\\
  \hline
  Work on most operating systems & The implementation depends on the OS\\
  \hline
  One part don't need to know the other &\\
  \hline
  No serialization/deserialization & \\
  \hline
  Very fast&\\
  \hline
\end{tabular}
~\\

The main problem is that the implementation depends on the OS, that is why we don't chose this communication.



\section{Pipe}

\begin{tabular}{|p{0.45\textwidth}||p{0.45\textwidth}|}
  \hline
  \textbf{Advantages}&\textbf{Drawback}\\
  \hline
                     &All POSIX systems, Windows\\
  \hline
                     &The implementation depends on the OS\\
  \hline
                     & Windows: Only one-way pipes (Anonymous pipe and Named pipe\cite{windows})\\
  \hline
\end{tabular}
~\\

This drawback is to penalizing, for the multiplatform constraint.

\section{Message parsing}

\begin{tabular}{|p{0.45\textwidth}||p{0.45\textwidth}|}
\hline
  \textbf{Advantages}&\textbf{Drawback}\\
\hline
&Used only in RPC, RMI, and MPI paradigms, Java RMI, CORBA, DDS, MSMQ, MailSlots, QNX, others\\
  \hline
                     &Slow and complex\\
  \hline
\end{tabular}
~\\

This drawback is to penalizing, for the multiplatform constraint.

% \section{File}

% \begin{tabular}{|p{0.45\textwidth}||p{0.45\textwidth}|}
% \hline
%   \textbf{Advantages}&\textbf{Drawback}\\
% \hline
% Problem when two software want to change the same file at the same moment& Communication asynchronous\\
% \hline
% \end{tabular}

% \section{Named pipe}

% \begin{tabular}{|p{0.45\textwidth}||p{0.45\textwidth}|}
% \hline
%   \textbf{Advantages}&\textbf{Drawback}\\
% \hline
% It is possible to use the Simulator outside the graphical modeling tool & Only available on POSIX systems, Windows, AmigaOS 2.0+\\
% \hline
% \end{tabular}

% \section{Socket}

% Details on the chapter \ref{chap:choice}.

% \section{Signal}

% \begin{tabular}{|p{0.45\textwidth}||p{0.45\textwidth}|}
% \hline
%   \textbf{Advantages}&\textbf{Drawback}\\
% \hline
%  & not usually used to transfer data\\
% \hline
% \end{tabular}


% \section{Shared Memory}

% \begin{tabular}{|p{0.45\textwidth}||p{0.45\textwidth}|}
% \hline
%   \textbf{Advantages}&\textbf{Drawback}\\
% \hline
% It is possible to use the Simulator outside the graphical modeling tool & \\
% \hline
% \end{tabular}




% \section{Our solution}

% For this project we chose to use socket enter the plugin and the simulator. So we need to create a layer of communication for the simulator and a layer of communication for the plugin. Both layer will listen on a thread.

% The solution was not in this list of common way to communicate inter process. In fact, we use the \textit{Runtime} class which is in the java library.~\\

% \noindent
% \begin{tabular}{|p{0.45\textwidth}||p{0.45\textwidth}|}
% \hline
%   \textbf{Advantages}&\textbf{Drawback}\\
% \hline
% It is possible to use the Simulator outside the graphical modeling tool & \\
% \hline
% Work with every type of simulator& \\
% \hline
% \end{tabular}

%%% Local Variables:
%%% mode: latex
%%% TeX-master: "../rapport_de_base"
%%% End:

% 
\chapter{SCCD}
\label{chap:sccd}

In this chapter, I will talk about SCCD. In the MSDL laboratory, they use SCCD to simulate Statechart models. After a presentation of my project in July, they suggested me to try their simulator and compare it to the Ciprian simulator. It is possible to find more explanation about SCCD on the Master's thesis published by Glenn De Jonghe \cite{sccd} and on the Master's thesis published by Joeri Exelmans \cite{sccd2}.

\section{Short description}

SCCD\footnote{signified StateChart and Class Diagram} is an open source simulator and code generator of Statechart developed by the MSDL laboratory. An SCCD model consists of a class Diagram and a StateChart attached on each classes. It also uses some special semantics to instantiate classes with an object manager.

Because a state machine is a statechart\footnote{Statechart is an extension of state machine, invented by David Harel}, for the rest of this chapter we can consider that our models conform to the SCCD language.


\section{Transformer}

As we saw before, the project has been written to have a ability to change the simulator. However, the model needs to be written in scxml standard to be interpreted by SCCD. So the first things that I have to do is a transformer in XSLT. XSLT is a language for transforming XML documents into other XML documents.
~\\

After some research on the internet I found only one transformer written by apache on Github\cite{apache}. The last commit of this project was in 2009, so we can assume that the project is abandoned. I had tried to use it but it didn't work. For this reason I have created a new transformer, but I used some part of this project.
~\\

My transformer has the ability to transform a xmi file as a scxml file. However, model given by my professor had some specificities so I take care to adapt to its.

For example, when there is a script in ABCD language and the script is <<send eventA to itsPinger>> the translation in the state machine is <<raise eventA to itsPinger>>. Then I use the fact that our project always have a \textit{SUS} class which contains all other classes, so in the scxml model the \textit{SUS} class start all other classes.

\section{Utilization}

\subsection{SCCD debugger}

In my project I want to visualize states of all state machine. To do that, I need some informations of the status of the model. SCCD don't permit this type of utilization, but the SCCD debugger written by Simon Van Mierlo can do it.
~\\

However, during my internship this debugger wasn't finished. The debugger can create only one class because the object manager doesn't work.
~\\

To prove that my scxml model created automatically will work when the debugger will be finished, I tried to do a proof of concept. %I use my scxml model created by my transformer automatically in SCCD and I verify the running.

\subsection{SCCD}

To do this prove of contest, I achieve some tests on the pure SCCD. I use the last version of SCCD published in the beginning of august.


I quickly realize that my model had infinite loop. These loops are explain because in the model there is some state which have transition on itself, and this transitions are always verify. In the Ciprian simulator it was not a problem because the user has to choose the next transition and so he was the condition. To fix this problem I add on these transitions a timer of 1s.

The Figure \ref{fig:sccd_resume} resume the work done to execute xmi model with SCCD. You can also see on the Figure \ref{fig:sccd} the result of the Ping-Pong example on SCCD. As it was expected there is an event which is send from ping to pong and then from pong to ping.
\begin{figure}[h]
  \centering
  \includegraphics[width=\linewidth]{scxml}
  \caption{Resume}
  \label{fig:sccd_resume}
\end{figure}

\begin{figure}[h]
  \centering
  \includegraphics[width=\linewidth]{sccd}
  \caption{pingpong simulation on SCCD}
  \label{fig:sccd}
\end{figure}


%%% Local Variables:
%%% mode: latex
%%% TeX-master: "../rapport_de_base"
%%% End:


\nocite{*}
%\input{annexe_}
\newpage
~\\
\newpage
 \listoffigures
 \printindex
 \bibliographystyle{plain}
  \bibliography{biblio}

\end{document}
%%%%%%%%%%%%%%%%% FIN DU DOCUMENT
%%% Local Variables:
%%% mode: latex
%%% TeX-master: t
%%% End:
