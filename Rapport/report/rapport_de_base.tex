\documentclass[a4paper, 11pt, oneside, oldfontcommands]{memoir}

%%%%% Packages %%%%%
\usepackage{lmodern}
\usepackage{palatino}
\usepackage[T1]{fontenc}
\usepackage[utf8]{inputenc}
\usepackage[english]{babel}


%%%%%%%%%%%%%%%%%%%%  PACKAGE SECONDAIRE

%\usepackage{amstext,amsmath,amssymb,amsfonts} % package math
%\usepackage{multirow,colortbl}	% to use multirow and ?
%\usepackage{xspace,varioref}
\usepackage[linktoc=all, hidelinks]{hyperref}			% permet d'utiliser les liens hyper textes
\usepackage{float}				% permet d ajouter d autre fonction au floatant
%\usepackage{wrapfig}			% permet d avoir des image avec texte coulant a cote
%\usepackage{fancyhdr}			% permet d inserer des choses en haut et en bas de chaque page
\usepackage{microtype}			% permet d ameliorer l apparence du texte
\usepackage[explicit]{titlesec}	% permet de modifier les titres
\usepackage{graphicx}			% permet d utiliser les graphiques
\graphicspath{{./images/}}		% to say where are image
%\usepackage{eso-pic} 			% to put figure in the background
\usepackage[svgnames]{xcolor}	% permet d avoir plus de 300 couleur predefini
%\usepackage{array}				% permet d ajouter des option dans les tableaux
%\usepackage{listings}			% permet d ajouter des ligne de code
%\usepackage{tikz}				% to draw figure
%\usepackage{appendix}			% permet de faire les index
%\usepackage{makeidx}			% permet de creer les index
%\usepackage{fancyvrb}			% to use Verbatim
%\usepackage{framed}				% permet de faire des environnement cadre
%\usepackage{fancybox}			% permet de realiser les cadres
\usepackage{titletoc}			% permet de modifier les titres
%\usepackage{caption}
\usepackage[a4paper, top=2cm, bottom=2cm]{geometry}
\usepackage{frbib}                      %permet d avoir une biblio francaise
\usepackage[babel=true]{csquotes}


\usepackage{graphicx}
\RequirePackage{pageGardeEnsta}	% permet d avoir la page de garde ensta

%\setcounter{secnumdepth}{2}		% permet d'augmenter la numerotation
%\setcounter{tocdepth}{2}		% permet d'augmenter la numerotation

%%%%%%%%%%%%%%%%%%  DEFINITION DES BOITES
\newcounter{rem}[chapter]

\newcommand{\remarque}[1]{\stepcounter{rem}\noindent\fcolorbox{OliveDrab}{white}{\parbox{\textwidth}{\textcolor{OliveDrab}{
\textbf{Remarque~\thechapter.\therem~:}}\\#1}}}

\newcounter{th}[chapter]

\newcommand{\theoreme}[2]{\noindent\fcolorbox{FireBrick}{white}{\stepcounter{th}
\parbox{\textwidth}{\textbf{\textcolor{FireBrick}{Théorème~\thechapter.\theth~:}}{\hfill \textit{#1}}\\#2}}}

\newcommand{\attention}[1]{\noindent\fcolorbox{white}{white}{\parbox{\textwidth}{\textcolor{FireBrick}{
\textbf{Attention !}}\\\textit{#1}\\}}}
%%%%%%%%%%%%%%%%%%%%%%%%%%%%%%%%%%%%%%%%%%%%%%%%%%%%%%%%%%%%%%%%%%%%%%%%%


%% INDEX %%%%%%%%%%%%%%%%%%%%%%%%%%%%%%%%%%%%%%%%%%%%%%%%%%%%
\makeindex

%%%%% Useful macros %%%%%
\newcommand{\latinloc}[1]{\ifx\undefined\lncs\relax\emph{#1}\else\textrm{#1}\fi\xspace}
\newcommand{\etc}{\latinloc{etc}}
\newcommand{\eg}{\latinloc{e.g.}}
\newcommand{\ie}{\latinloc{i.e.}}
\newcommand{\cad}{c'est-à-dire }
\newcommand{\st}{\ensuremath{\text{\xspace s.t.\xspace}}}

%%%% Definition des couleur %%%%

\newcommand\couleurb[1]{\textcolor{SteelBlue}{#1}}
\newcommand\couleurr[1]{\textcolor{DarkRed}{#1}}


%% number page style style %%%%%%%%%%%%%%%%%%%%%%%%%%%%%%%%%%%%%%%%%%%%%%%%%%%%%%

\pagestyle{plain}
%\pagestyle{empty}
%\pagestyle{headings}
%\pagestyle{myheadings}



%% chapters style %%%%%%%%%%%%%%%%%%%%%%%%%%%%%%%%%%%%%%%%%%%%%%%%%%%%%%
%% You may try several styles (see more in the memoir manual).

%\chapterstyle{veelo}
%\chapterstyle{chappell}
%\chapterstyle{ell}
%\chapterstyle{ger}
%\chapterstyle{pedersen}
%\chapterstyle{verville}
\chapterstyle{madsen}
%\chapterstyle{thatcher}


%%%%% Report Title %%%%%
\title{Simulation of StateChart}
\author{\textsc{Rigaud Michaël}}
\date{\today}
\doctype{Rapport de stage}
\promo{promo 2017}
\etablissement{\textsc{Ensta} Bretagne\\2, rue François Verny\\
  29806 \textsc{Brest} cedex\\\textsc{France}\\Tel +33 (0)2 98 34 88 00\\ \url{www.ensta-bretagne.fr}}
\logoEcole{\includegraphics[height=4.2cm]{logo_ENSTA_Bretagne_Vertical_CMJN}}



%%%%%%%%%%%%%%%%%% DEBUT DU DOCUMENT
\begin{document}

\maketitle
\thispagestyle{empty}
\newpage

\tableofcontents


%%%%%%%%%%%%%%%%% INTRODUCTION

\chapter*{Introduction}
\addcontentsline{toc}{chapter}{Introduction}
\newpage	  
%%%%%%%%%%%%%%%%%%%%%%%%

\part{Presentation}



\chapter{Presentation of the project}

\section{The goal}

The goal of this project is to create a simulator of Statechart which can be use with UMLDesigner. This simulator should permit to visualize and debug a model of a state machine. Moreover, UMLDesigner is a modeling software for UML model and Statechart, so we could create the model and simulate it on the same tools. The picture \ref{fig:project} represent the aim of this project.

\begin{figure}[h]
  \centering
\includegraphics[width=\textwidth]{project}  
  \caption{Description of the project}
  \label{fig:project}
\end{figure}


\section{Tools at the disposal}

At the begin of this project, some of the tools, which were needed, existed. In fact, ULMDesigner is a UML modeling tool develop by \textit{Obeo}. However, it didn't exist yet a simulator for Statechart adapted for UMLDesigner. On the chapter \ref{chap:UMLDesigner}, the running of UMLDesigner will be discuss.~\\

Then, Mr Ciprian Theodorov, one of my professor, has developed a simulator for Statechart. This simulator needed to be improved, but it composed a good beginning for this project. 



%%% Local Variables: 
%%% mode: latex
%%% TeX-master: "../rapport_de_base"
%%% End: 

\chapter{\umld}
\label{chap:UMLDesigner}

\section{Description}

\umld is an open-source tool to edit and visualize UML2 models created by the French company:
\textit{Obeo}. The project is licensed under the EPL\footnote{Eclipse public license}

\begin{figure}[h] \centering
  \includegraphics[width=0.3\textwidth]{logo}
  \caption{UML Designer logo}
  \label{fig:logo}
\end{figure}

\section{Utilization}

\umld is a graphical modeling tool for UML2 as defined by OMG\footnote{Object Management Group\cite{omg}}. As
you can see on the figure \ref{fig:umldesigner}, it permit to create diagram on which ones it is
possible to add some elements. The type of the elements proposed depend on the types of the diagram
chosen. For example, if you choose a \textit{User case diagram} it is possible to add 'user'
component that is impossible in \textit{Class diagram}.

So with graphical action it is possible to create many UML diagram which have transverse elements.

To finish, it is possible to create the code of the application that you have develop from the
model.


\begin{figure}[h] \centering
  \includegraphics[width=0.8\textwidth]{umldesigner}
  \caption{Screen shot of \umld}
  \label{fig:umldesigner}
\end{figure}



\section{List of diagram supported}

\begin{itemize}
\item Packages diagram
\item Use case diagram
\item Activity diagram
\item Class diagram
\item Component diagram
\item Composite Structure diagram
\item Sequence diagram
\item State Machine diagram
\item Documentation table
\item Use Case cross table
\item Package containment diagram
\item Profile diagram
\end{itemize}



\section{Released}
\begin{center}

  \begin{tabular}{|c|c|}
    \hline
    \textbf{Version} & \textbf{Release Date} \\
    \hline
    1.0.0 &2012\\
    \hline
    2.0.0 &17 January 2013\\
    \hline
    2.1.0 &1 February 2013\\
    \hline
    2.2.0 &12 April 2013\\
    \hline
    2.3.0 &13 June 2013\\
    \hline
    2.4.0 &13 September 2013\\
    \hline
    3.0.0 &17 January 2014\\
    \hline
    4.0.0 &8 July 2014\\
    \hline
    4.0.1 &5 August 2014\\
    \hline
    5.0.0 &29 May 2015\\
    \hline
    \cellcolor{green}6.0.0 &19 October 2015\\
    \hline
  \end{tabular}

  \begin{tabular}{c}
    Legend:\\
    \cellcolor{green}Latest stable release\\
  \end{tabular}

\end{center}

\section{Base on}

\umld is based on a Eclipse and Sirius. It is a UML2 Eclipse plugin.

\subsection{Sirius}
Sirius is an open-source software project of the Eclipse Foundation. Sirius allows to create graphical modeling workbench. It include EMF\footnote{Eclipse Modeling Framework} and GMF\footnote{Graphical Modeling Framework}. On the figure \ref{fig:sirius}, it is possible to see the architecture of Sirius.


\begin{figure}[h]
  \centering
  \includegraphics[width=\textwidth]{sirius_archi}
  \caption{Sirius architecture\cite{sirius}}
  \label{fig:sirius}
\end{figure}

\subsection{Eclipse}

\umld is base on Eclipse.
The interface is the same as Eclipse. You can notice on figure
\ref{fig:umldesigner} that the menu are the same in the both software.



\begin{figure}[h] \centering
  \includegraphics[width=0.5\textwidth]{archi}
  \caption{The \umld kernel}
  \label{fig:kernel}
\end{figure}






%%% Local Variables:
%%% mode: latex
%%% TeX-master: "../rapport_de_base"
%%% End:


\chapter{Simulator}

\section{Description}


At the beginning of this project, we had at our disposal the simulator of Mr Teodorov (figure \ref{fig:sim}). This simulator have a graphic user interface as you can see on the figure \ref{fig:sim}.


\begin{figure}[h]
  \centering
  \includegraphics[width=0.5\textwidth]{simulator}
  \caption{Mr Teodorov simulator}
  \label{fig:sim}
\end{figure}


The simulator is compose on 4 part.
\begin{itemize}
\item On the top: some buttons to select an action
\item On the top-left-corner: The list of the next step
\item On the bottom-left-corner: The State Machine associated to the Current State.
\item On the right: A visualization of the Statechart
\end{itemize}

\section{Specificity of the uml file}

This simulator simulate a uml file. The uml file need to have a particular architecture.

\umld to save the uml project use 2 files. The first is named ``model.uml'' and the second is named ``representation.aird''.

To work, the simulator need the \textit{model.uml} file. Moreover, this file need to contain some specifics feature. It need a class \textbf{SUS} which contain the declaration of all other classes and all other classes need to have a State Machine diagram associated. You can see on the figure \ref{fig:simulateur}, that all classes need to have their own State Machine diagrams.

\begin{figure}[h!]
  \centering
  \includegraphics[width=\textwidth]{simulation}
  \caption{representation of the most important elements of the simulator}
  \label{fig:simulateur}
\end{figure}




%%% Local Variables:
%%% mode: latex
%%% TeX-master: "../rapport_de_base"
%%% End:


\part{Study of the subject}


\chapter{Communication inter process}

\section{Type of communication conceivable}

A lot of type of communication inter process were suggested to create a discussion enter the plugin and the simulator. But we will present only the most consistent.

The communication is the part the most important of this project, because that will implement the interface between the two software.

\subsection{Socket}

\begin{tabular}{|p{0.45\textwidth}||p{0.45\textwidth}|}
\hline
  \textbf{Advantages}&\textbf{Drawback}\\
\hline
Work with every simulator type (python, java, ...) & communication synchronous\\
\hline
\end{tabular}

\subsection{File}

\begin{tabular}{|p{0.45\textwidth}||p{0.45\textwidth}|}
\hline
  \textbf{Advantages}&\textbf{Drawback}\\
\hline
Problem when two software want to change the same file at the same moment& Communication asynchronous\\
\hline
\end{tabular}

\subsection{Named pipe}

\begin{tabular}{|p{0.45\textwidth}||p{0.45\textwidth}|}
\hline
  \textbf{Advantages}&\textbf{Drawback}\\
\hline
It is possible to use the Simulator outside the graphical modeling tool & \\
\hline
\end{tabular}


\subsection{Shared Memory}

\begin{tabular}{|p{0.45\textwidth}||p{0.45\textwidth}|}
\hline
  \textbf{Advantages}&\textbf{Drawback}\\
\hline
It is possible to use the Simulator outside the graphical modeling tool & \\
\hline
\end{tabular}


\subsection{Thread}

\begin{tabular}{|p{0.45\textwidth}||p{0.45\textwidth}|}
\hline
  \textbf{Advantages}&\textbf{Drawback}\\
\hline
&Need to recode the simulator \\
\hline
\end{tabular}


\subsection{Our solution}

The solution was not in this list of common way to communicate inter process. In fact, we use the \textit{Runtime} class which is in the java library.~\\


\begin{tabular}{|p{0.45\textwidth}||p{0.45\textwidth}|}
\hline
  \textbf{Advantages}&\textbf{Drawback}\\
\hline
It is possible to use the Simulator outside the graphical modeling tool & \\
\hline
Work with every type of simulator& \\
\hline
\end{tabular}





%%% Local Variables: 
%%% mode: latex
%%% TeX-master: "../rapport_de_base"
%%% End: 




%%%% CONCLUSION %%%%%%%%%

\chapter*{Conclusion}
\addcontentsline{toc}{chapter}{Conclusion}
\newpage

%%%% ANNEXE %%%%%%%%%%%%

\part*{Annexe}

\chapter{Organisation of the work}

\section{Calendar}


\begin{tabular*}{1\textwidth}{@{\extracolsep{\fill}} |c|*{14}{c|}}
\hline
  Tasks/weeks & 1 &2 &3&4&5&6&7&8&9&10&11&12&13&14\\
\hline
State of the art&-&-&&&&&&&&&&&&\\
\hline
Create a plugin&&&-&&&&&&&&&&&\\
\hline
Visualize the simulation&&&&-&-&-&&&&&&&&\\
\hline
Unit tests&&&&&&-&-&&&&&&&\\
\hline
Integration tests&&&&&&&&-&&&&&&\\
\hline
Improve the simulator&&&&&&&&&-&-&&&&\\
\hline
Try an other simulator&&&&&&&&&&&-&-&&\\
\hline
Redaction&&-&-&-&-&-&-&-&-&-&-&-&-&\\
\hline
Soutenance&&&&&&&&&&&&&&-\\
\hline
\end{tabular*}

\section{Tools use for the project}

The Framaboard application:

\begin{figure}[h]
  \centering
  \includegraphics[width=\textwidth]{framaboard}
  \caption{Screen shot of the framaboard}
  \label{fig:framaboard}
\end{figure}


The web site of MSDL researcher:

\begin{figure}[h]
  \centering
  \includegraphics[width=\textwidth]{msdl}
  \caption{MSDL web site}
  \label{fig:msdl}
\end{figure}


%%% Local Variables: 
%%% mode: latex
%%% TeX-master: "../rapport_de_base"
%%% End: 


\appendix
\nocite{*}
%\input{annexe_}
\newpage
 \listoffigures
 \printindex
 \bibliographystyle{plain}
  \bibliography{biblio}

\end{document}
%%%%%%%%%%%%%%%%% FIN DU DOCUMENT