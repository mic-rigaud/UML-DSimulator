
\chapter{Communication enter process}
\label{annex:choice}


\section{File}

\begin{tabular}{|p{0.45\textwidth}||p{0.45\textwidth}|}
\hline
  \textbf{Advantages}&\textbf{Drawback}\\
\hline
Problem when two software want to change the same file at the same moment& Communication asynchronous\\
\hline
\end{tabular}

\section{Named pipe}

\begin{tabular}{|p{0.45\textwidth}||p{0.45\textwidth}|}
\hline
  \textbf{Advantages}&\textbf{Drawback}\\
\hline
It is possible to use the Simulator outside the graphical modeling tool & Only available on POSIX systems, Windows, AmigaOS 2.0+\\
\hline
\end{tabular}

\section{Socket}

Details on the chapter \ref{chap:choice}.

\section{Signal}

\begin{tabular}{|p{0.45\textwidth}||p{0.45\textwidth}|}
\hline
  \textbf{Advantages}&\textbf{Drawback}\\
\hline
 & not usually used to transfer data\\
\hline
\end{tabular}


\section{Shared Memory}

\begin{tabular}{|p{0.45\textwidth}||p{0.45\textwidth}|}
\hline
  \textbf{Advantages}&\textbf{Drawback}\\
\hline
It is possible to use the Simulator outside the graphical modeling tool & \\
\hline
\end{tabular}




% \section{Our solution}

% For this project we chose to use socket enter the plugin and the simulator. So we need to create a layer of communication for the simulator and a layer of communication for the plugin. Both layer will listen on a thread.

% The solution was not in this list of common way to communicate inter process. In fact, we use the \textit{Runtime} class which is in the java library.~\\

% \noindent
% \begin{tabular}{|p{0.45\textwidth}||p{0.45\textwidth}|}
% \hline
%   \textbf{Advantages}&\textbf{Drawback}\\
% \hline
% It is possible to use the Simulator outside the graphical modeling tool & \\
% \hline
% Work with every type of simulator& \\
% \hline
% \end{tabular}

%%% Local Variables:
%%% mode: latex
%%% TeX-master: "../rapport_de_base"
%%% End:
