
\chapter{Simulator}

\section{Description}


At the beginning of this project, we had at our disposal the simulator of Mr Teodorov (figure \ref{fig:sim}). This simulator have a graphic user interface as you can see on the figure \ref{fig:sim}.


\begin{figure}[h]
  \centering
  \includegraphics[width=0.5\textwidth]{simulator}
  \caption{Mr Teodorov simulator}
  \label{fig:sim}
\end{figure}


The simulator is compose on 4 part.
\begin{itemize}
\item On the top: some buttons to select an action
\item On the top-left-corner: The list of the next step
\item On the bottom-left-corner: The State Machine associated to the Current State.
\item On the right: A visualization of the Statechart
\end{itemize}

\section{Specificity of the uml file}

This simulator simulate a uml file. The uml file need to have a particular architecture.

\umld to save the uml project use 2 files. The first is named ``model.uml'' and the second is named ``representation.aird''.

To work, the simulator need the \textit{model.uml} file. Moreover, this file need to contain some specifics feature. It need a class \textbf{SUS} which contain the declaration of all other classes and all other classes need to have a State Machine diagram associated. You can see on the figure \ref{fig:simulateur}, that all classes need to have their own State Machine diagrams.

\begin{figure}[h!]
  \centering
  \includegraphics[width=\textwidth]{simulation}
  \caption{representation of the most important elements of the simulator}
  \label{fig:simulateur}
\end{figure}




%%% Local Variables:
%%% mode: latex
%%% TeX-master: "../rapport_de_base"
%%% End:
