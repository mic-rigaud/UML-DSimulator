
\chapter{Professional and personal balance sheet}

\section{Profits from this experience}

\subsection{The acquisition of new technical skills}

During this internship I have discovered new technical skills, and I have consolidated skills acquired at ENSTA Bretagne. My project was developed in Java, and I choose to use a managing tools Framaboard (to create Kanban board) and a version control system Git. I understand better why it is important to use this type of tools. Framaboard was very important to keep an overview of my target and keep in mind the task that I have to implement after. Git was very useful to keep track of my work.

Moreover, my work has allowed me to better understand Statechart. In fact, my only experience at ENSTA Bretagne with Statechart was through UML, but during my internship I follow some presentation about Statechart and I try to use SCCD which is a simulator of Statechart.



\subsection{The discovery of a research center}

During this internship I discover the operation of a research center. Even if I will work for the DGA\footnote{Direction générale de l'armement} which is different of a research center, It was very interesting to discover how it works.

I had the opportunity to see some Master Thesis presentation, and follow an introduction of all MSDL researches. It gave me the possibility to better understand their research on Modeling and why they are important.


\section{Encountered difficulties}

During this internship I meet one of the most common problem in software development: the problem of integration. In fact, for me it was very difficult to find a API\footnote{Application Programming Interface} to do action on \umld. In fact, \umld is based on Sirius, EMF, and the Eclipse Kernel, so I had to understand their API to interact on the user interface of \umld. It was very difficult, because when I wanted to do something I had first to find on which API I had to be connected and then how use it.
~\\

An other problem was that the simulator given by Mr Teodorov was stable but not finished. In fact, this simulator has some constraints which block the advanced of new abilities. For example, This simulator doesn't take care about instances. It is a real problem because in OOP\footnote{Object-Oriented programming} there is often object which are instantiated several times. Even if, my plugin was written to have the possibility to show multiple instances, it was not possible to test it on real model.



%%% Local Variables:
%%% mode: latex
%%% TeX-master: "../rapport_de_base"
%%% End:
