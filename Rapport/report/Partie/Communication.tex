
\chapter{Communication inter process}

\section{Type of communication conceivable}

A lot of type of communication inter process were suggested to create a discussion enter the plugin and the simulator. But we will present only the most consistent.

The communication is the part the most important of this project, because that will implement the interface between the two software.

\subsection{Socket}

\begin{tabular}{|p{0.45\textwidth}||p{0.45\textwidth}|}
\hline
  \textbf{Advantages}&\textbf{Drawback}\\
\hline
Work with every simulator type (python, java, ...) & communication synchronous\\
\hline
\end{tabular}

\subsection{File}

\begin{tabular}{|p{0.45\textwidth}||p{0.45\textwidth}|}
\hline
  \textbf{Advantages}&\textbf{Drawback}\\
\hline
Problem when two software want to change the same file at the same moment& Communication asynchronous\\
\hline
\end{tabular}

\subsection{Named pipe}

\begin{tabular}{|p{0.45\textwidth}||p{0.45\textwidth}|}
\hline
  \textbf{Advantages}&\textbf{Drawback}\\
\hline
It is possible to use the Simulator outside the graphical modeling tool & \\
\hline
\end{tabular}


\subsection{Shared Memory}

\begin{tabular}{|p{0.45\textwidth}||p{0.45\textwidth}|}
\hline
  \textbf{Advantages}&\textbf{Drawback}\\
\hline
It is possible to use the Simulator outside the graphical modeling tool & \\
\hline
\end{tabular}


\subsection{Thread}

\begin{tabular}{|p{0.45\textwidth}||p{0.45\textwidth}|}
\hline
  \textbf{Advantages}&\textbf{Drawback}\\
\hline
&Need to recode the simulator \\
\hline
\end{tabular}


\subsection{Our solution}

The solution was not in this list of common way to communicate inter process. In fact, we use the \textit{Runtime} class which is in the java library.~\\


\begin{tabular}{|p{0.45\textwidth}||p{0.45\textwidth}|}
\hline
  \textbf{Advantages}&\textbf{Drawback}\\
\hline
It is possible to use the Simulator outside the graphical modeling tool & \\
\hline
Work with every type of simulator& \\
\hline
\end{tabular}





%%% Local Variables: 
%%% mode: latex
%%% TeX-master: "../rapport_de_base"
%%% End: 
