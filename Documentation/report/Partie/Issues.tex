
\chapter{Goals}
\label{chap:issues}


As it was explained in the introduction, the aim of this project is to propose a visualization of UML Model for \umld.

\section{Short term goals}

In the short term, this plugin should permit Mr. Champeau and Mr. Teodorov to use \umld in the classroom, as a free and portable alternative to Rhapsody. In this way, every student will be able to install on their own computer the tool use in classroom.

Advantages of \umld when compared to Rhapsody: free, open source, works on Windows, Linux, and Apple.

\section{Long term goals}

Because this plugin is open source and downloadable on a Git Server\footnote{\url{https://msdl.uantwerpen.be/git/michael/UML-DSimulator}}, it can be used by anybody. My hope is that this plugin can be improved by the community, student, teacher, \etc... and becomes a serious alternative to Rhapsody and Papyrus simulator.

~\\

  \begin{figure}[h]
    \begin{minipage}{0.45\linewidth}
      \centering
      \includegraphics[width=0.7\textwidth]{Rapsody}
      \caption{Rational Rhapsody}
      \label{fig:rhapsody}
    \end{minipage}\hfill
    \begin{minipage}{0.45\linewidth}
      \centering
      \includegraphics[width=\textwidth]{Papyrus}
      \caption{Papyrus}
      \label{fig:papyrus}
    \end{minipage}
  \end{figure}


%%% Local Variables:
%%% mode: latex
%%% TeX-master: "../rapport_de_base"
%%% End:
